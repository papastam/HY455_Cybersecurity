\documentclass{article}
\usepackage{graphicx} % Required for inserting images
\usepackage{amsmath}
% \usepackage{cancel}
% \usepackage{enumitem}
\usepackage{tabularx}

\title{HY111, Assignment 1}
\author{Chris Papastamos (csd4569)}

\begin{document}

\maketitle

\section*{Exercise 1}
\begin{enumerate}
    \item $|\vec{v}|= \sqrt{{3^2}+{-1}^2} = \sqrt{10}$ 
    \item $|\vec{u}|= \sqrt{{3^2}+{2}^2} = \sqrt{13}$ 
    \item $\vec{v}\cdot\vec{u}= 3*0 + -1*3 + 0*2 = -3$
    \item $\vec{u}\times\vec{v} =  \begin{vmatrix}
                                i & j & k\\
                                0 & 3 & 2\\
                                3 & -1 & 0 
                                \end{vmatrix}
                                 = \begin{vmatrix}
                                3 & 2\\
                                -1 & 0 
                                \end{vmatrix}i
                                -
                                \begin{vmatrix}
                                0 & 2\\
                                3 & 0 
                                \end{vmatrix}j
                                +
                                \begin{vmatrix}
                                0 & 3\\
                                3 & -1 
                                \end{vmatrix}k
                                =2i+6j-9k
                                $

    \item $\vec{v}\times\vec{u} =  \begin{vmatrix}
                                i & j & k\\
                                3 & -1 & 0\\
                                0 & 3 & 2
                                \end{vmatrix}
                                 = \begin{vmatrix}
                                -1 & 0\\ 
                                3 & 2
                                \end{vmatrix}i
                                -
                                \begin{vmatrix}
                                3 & 0\\
                                0 & 2
                                \end{vmatrix}j
                                +
                                \begin{vmatrix}
                                3 & -1\\ 
                                0 & 3
                                \end{vmatrix}k
                                =-2i-6j+9k                                
                                $

    \item $|\vec{v}\times\vec{u}|=\sqrt{(-2)^2+(-6)^2+(9)^2}= \sqrt{4+36+81} = \sqrt{121}$
    \item $\theta=\cos^{-1}({\frac{\vec{u}\cdot\vec{v}}{|\vec{u}||\vec{v}|}})= \cos^{-1}(\frac{-3}{\sqrt{10}\sqrt{13}}) \approx 1,83 rad$
    \item $proj_{\vec{v}} \vec{u} = (\frac{\vec{u}\vec{v}}{|\vec{v}|^2})\vec{v} = (\frac{-3}{10})\vec{v} = -0,3\vec{v} = -0,9i+0,3j$

\end{enumerate}
% The above calculations in GeoGebra

\begin{center}
    % \includegraphics[scale = 0.25]{ex1_geogebra.png}
\end{center}

\section*{Exercise 2}
$\epsilon_1 = 4x+3y-2 \xrightarrow{} \vec{v_1} = (3,-4)$\\
$\epsilon_1 = 5x-2y-3 \xrightarrow{} \vec{v_2} = (-2,-5)$
\\
\\
$\theta=cos^{-1}(\frac{\vec{v_1}\vec{v_2}}{|\vec{v_1}||\vec{v_2}|}) = cos^{-1}(\frac{(-2*3)+((-4)*(-5))}{\sqrt{3^2+(-4)^2}\sqrt{(-2)^2+(-5^2)}}) = cos^{-1}(\frac{14}{\sqrt{25}\sqrt{29}}) = cos^{-1}(\frac{14}{5\sqrt{29}}) \approx$ 1,02 rad

\begin{center}
    % \includegraphics[scale = 0.25]{ex2_geogebra.png}
\end{center}

\section*{Exercise 3}
$\vec{v}= (1,1)$\\
$\vec{u}= (1,2)$\\
$\vec{w}= (x,y)$
\\
$proj_{\vec{v}}\vec{w}= proj_{\vec{v}}\vec{u} \Rightarrow
% (\frac{\vec{v}\vec{w}}{\cancel{|\vec{v}|^2}})\cancel{\vec{v}}=(\frac{\vec{v}\vec{u}}{\cancel{|\vec{v}|^2}})\cancel{\vec{v}} \Rightarrow
\vec{w}\vec{v}=\vec{u}\vec{v}\Rightarrow
x+y=1+2\Rightarrow
x+y=3
$

\begin{center}
    % \includegraphics[scale = 0.25]{ex3_geogebra.png}
\end{center}

\section*{Exercise 4}
\begin{enumerate}
    \item $(\vec{u} i)^2+(\vec{u}j)^2+(\vec{u}k)^2=|\vec{u}|^2 \Rightarrow
    u_i^2+ u_j^2+u_k^2 = \sqrt{u_i^2+ u_j^2+u_k^2}^2 \Rightarrow
    u_i^2+ u_j^2+u_k^2 = u_i^2+ u_j^2+u_k^2
    $. From the above the statement is \textbf{True}
    \\
    % \item Lets assume the three vectors: $\vec{u}=(1,0,0), \vec{v}=(0,1,0), \vec{w}=(2,0,0)$. We observe that although $\vec{u} \perp \vec{v}$ and $\vec{v} \perp \vec{w}$ it is clear that $\vec{u} \cancel{\perp} \vec{w}$. Therefore the statement is \textbf{False}
    \\
    \item This statement is \textbf{False} because the perpendicular vectors to the vector (1,1,1) belong to the same plane (not the same line)
    \\
    \item This statement is \textbf{True} because the inner product of tow perpendicular vectors is 0, witch zeroes out the projection as well
    \\
    \item This statement is \textbf{False} because $\vec{u} \times \vec{v} = 0$ when $\vec{u} \parallel \vec{v}$. That means that $ \vec{u} \parallel \vec{u} \times \vec{v}$ which is clearly not true
    \\
    % \item $proj_{\vec{v}}\vec{u} = proj_{\vec{u}}\vec{v} \Rightarrow (\frac{\cancel{\vec{u}\vec{v}}}{|\vec{v}|^2})\vec{v} = (\frac{\cancel{\vec{v}\vec{u}}}{|\vec{u}|^2})\vec{u} \Rightarrow \frac{\vec{v}}{|\vec{v}|^2} = \frac{\vec{u}}{|\vec{u}|^2}$\\\\
    Lets assume the following vectors: $\vec{v}=(1,0,0)$, $\vec{u}=(1,1,0)$:
    $$\frac{(1,0,0)}{1} \neq \frac{(1,1,0)}{\sqrt{2}}$$
\end{enumerate}

\section*{Exercise 5}
To find the distance between the plane and the line, we need to first find a point on the line and the normal vector to the plane. 
\\\\
We can see that when t = 0, the parametric equations of the line give the point $P=(2, 1, -\frac{1}{2})$. So, we can use this point as a point on the line
\\\\
The coefficients of x, y, and z in the equation of the plane x + 2y + 6z = 6 give us the normal vector to the plane, which is $\vec{u}=(1, 2, 6)$
\\\\
Now, we can use the formula for the distance between a point and a plane:

$$distance = |\frac{ax + by + cz - d}{\sqrt{a^2 + b^2 + c^2}}| = |\frac{1(2) + 2(1) + 6(-0.5) - 6}{\sqrt{1^2 + 2^2 + 6^2}}| = |-\frac{1}{3}| = \frac{1}{3}$$

Therefore, the distance between the plane and the line is $\frac{1}{3}$ 

\section*{Exercise 6}
For checking if the line is parallel to the plane it is enough if one vector of the plane is parallel to the vector described by the line's parametric equation
\\
\\
\begin{tabularx}{\textwidth}{X | X}
    Line:   & Plane:    \\ 
    x=3-2t  & 2x+y-z=17 \\
    y=1+5t              \\
    z=-2-3t             \\
                        \\
                        \\
    The vector from point (0,0,0) is: &Lets take a vector on the plane: \\
    $\vec{v}=2i+5j-3k$  & $\vec{u}=2i+j-k$    
\end{tabularx}
\\
\\
$$\vec{u}\cdot\vec{v} = 2*2+5*1+(-3)*(-1)= 12 $$
% The inner product of the two vectors is $\vec{u} \times \vec{v} \neq 0$ which means that $\vec{u} \cancel{\parallel} \vec{v}$
\newpage
\section*{Exercise 7}
First, we find the normal vector to the given plane by taking the coefficients of x, y, and z as the components of the vector. The normal vector is $\vec{u} =(2, 3, 1)$
\\\\
Since the plane we want to find is perpendicular to the given plane, its normal vector must be parallel to the given plane's normal vector. Therefore, we can take the cross product of the normal vector to the given plane with any other vector that is not parallel to it, such as the vector $\vec{v}= (1, 0, 0)$, to obtain a vector that is perpendicular to both the given plane's normal vector and the vector we chose.

$$\vec{u} \times \vec{v} = (2, 3, 1) \times (1, 0, 0) = (0, 1, -3)$$

Now we have two vectors that are perpendicular to the plane we want to find: $\vec{v}=(1, 0, 0)$ and $\vec{u}\times \vec{v} =(0, 1, -3)$. We can use these two vectors as a basis for the plane and find its equation by using the point-normal form.
The normal vector to the plane is $\vec{u}\times \vec{v}  =(0, 1, -3)$. Therefore, the equation of the plane is:

$$0(x - 1) + 1(y - 1) - 3(z - 1) = 0 \Rightarrow y - 3z + 2 = 0$$

\begin{center}
    % \includegraphics[scale = 0.25]{ex7_geogebra.png}
\end{center}
\end{document}